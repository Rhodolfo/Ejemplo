\section{Initializando el repositorio}

Vamos a inizializar el repositorio

Ir a la p\'{a}gina de GitHub, crear un nuevo repositorio.
Despu\'{e}s podemos inicializar una copia local. 

> git init
Initialized empty Git repository in /home/rho/Documents/GitHub/.git/
>git remote add origin git@github.com:Rhodolfo/Ejemplo.git
>git remote -v
origin	git@github.com:Rhodolfo/Ejemplo.git (fetch)
origin	git@github.com:Rhodolfo/Ejemplo.git (push)

Este mensaje simplemente nos dice que el repositorio local se ha creado correctamente
y que est\'{a} atado al repositorio remoto.

Alternativamente, podemos clonar un repositorio,
> git clone git@github.com:Rhodolfo/Ejemplo.git
Cloning into 'Ejemplo'...
remote: Counting objects: 3, done.
remote: Total 3 (delta 0), reused 0 (delta 0), pack-reused 0
Receiving objects: 100\% (3/3), done.
Checking connectivity... done.

\section{Guardando Cambios}

En el repositorio empec\'{e} a escribir estas notas,
al hacer un cambio puedo simplemente en que diferimos el local y el remoto.

> git status
On branch master
Changes not staged for commit:
  (use "git add <file>..." to update what will be committed)
  (use "git checkout -- <file>..." to discard changes in working directory)

	modified:   README.md

Untracked files:
  (use "git add <file>..." to include in what will be committed)

	notas.tex

no changes added to commit (use "git add" and/or "git commit -a")

Git nos dice que se ha cambiado README.md,
tambi\'{e} indica que hemos creado un nuevo archivo notas.tex.
Podemos pedirle a git que guarde estos cambios,

> git add README.md
> git add notas.tex
> git status






