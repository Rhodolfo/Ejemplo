\section{Initializando el repositorio}

Vamos a inizializar el repositorio

Ir a la p\'{a}gina de GitHub, crear un nuevo repositorio.
Despu\'{e}s podemos inicializar una copia local. 

> git init
Initialized empty Git repository in /home/rho/Documents/GitHub/.git/
>git remote add origin git@github.com:Rhodolfo/Ejemplo.git
>git remote -v
origin	git@github.com:Rhodolfo/Ejemplo.git (fetch)
origin	git@github.com:Rhodolfo/Ejemplo.git (push)

Este mensaje simplemente nos dice que el repositorio local se ha creado correctamente
y que est\'{a} atado al repositorio remoto.

Alternativamente, podemos clonar un repositorio,
> git clone git@github.com:Rhodolfo/Ejemplo.git
Cloning into 'Ejemplo'...
remote: Counting objects: 3, done.
remote: Total 3 (delta 0), reused 0 (delta 0), pack-reused 0
Receiving objects: 100\% (3/3), done.
Checking connectivity... done.

\section{Guardando Cambios}

En el repositorio empec\'{e} a escribir estas notas,
al hacer un cambio puedo simplemente en que diferimos el local y el remoto.

> git status
On branch master
Changes not staged for commit:
  (use "git add <file>..." to update what will be committed)
  (use "git checkout -- <file>..." to discard changes in working directory)

	modified:   README.md

Untracked files:
  (use "git add <file>..." to include in what will be committed)

	notas.tex

no changes added to commit (use "git add" and/or "git commit -a")

Git nos dice que se ha cambiado README.md,
tambi\'{e} indica que hemos creado un nuevo archivo notas.tex.
Podemos pedirle a git que guarde estos cambios,

> git add README.md
> git add notas.tex
> git status
On branch master
Changes to be committed:
  (use "git reset HEAD <file>..." to unstage)

	modified:   README.md
	new file:   notas.tex

Ahora tiene sentido sincronizar nuestros cambios con el remoto,

git commit -m "Cambio en README.md, nuevo archivo notas.tex"
[master 1626734] Cambio en README.md, nuevo archivo notas.tex
 2 files changed, 59 insertions(+)
 create mode 100644 notas.tex

Si estamos contentos,

> git push origin master
Counting objects: 4, done.
Delta compression using up to 8 threads.
Compressing objects: 100\% (3/3), done.
Writing objects: 100\% (4/4), 1.09 KiB | 0 bytes/s, done.
Total 4 (delta 0), reused 0 (delta 0)
To git@github.com:Rhodolfo/Ejemplo.git
   67c1f32..1626734  master -> master

> git status
On branch master
Your branch is up-to-date with 'origin/master'.
nothing to commit, working directory clean



\section{Ramas, Colaboraci\'{o}n y Conflictos}

Normalmente uno no cambia la el repositorio principal del proyecto,
conviene crear una 'rama,' las ramas son copias de un repositorio. 

$ git branch uno
M       notas.tex
Switched to branch 'uno'

$ git add README.md
$ git commit -m "Cambio en rama uno"
[uno 6fbb706] Cambio en rama uno
 1 file changed, 2 insertions(+)

$ git push origin uno
Counting objects: 3, done.
Delta compression using up to 8 threads.
Compressing objects: 100% (3/3), done.
Writing objects: 100% (3/3), 346 bytes | 0 bytes/s, done.
Total 3 (delta 0), reused 0 (delta 0)
To git@github.com:Rhodolfo/Ejemplo.git
 * [new branch]      uno -> uno


Digamos que existe una rama llamada 'dos',
donde tamb'{e}n se ha cambiado el README.

Podemos hacer un merge, el cu\'{a}l
junta los cambios y pide al usuario que resuelva conflictos.
Trabajando desde la rama dos:

$ git merge uno
Auto-merging README.md
CONFLICT (content): Merge conflict in README.md
Automatic merge failed; fix conflicts and then commit the result.

Tenemos un conflicto en el archivo README.md entre las ramas uno y dos.

$ git status 
On branch dos
Your branch is up-to-date with 'origin/dos'.
You have unmerged paths.
  (fix conflicts and run "git commit")

Unmerged paths:
  (use "git add <file>..." to mark resolution)

        both modified:   README.md

no changes added to commit (use "git add" and/or "git commit -a")

Tenemos que entrar a editar README.md, 
en este caso tendr\'{a} los cambios de las ramas uno y dos,
escojeremos a mano como queremos que se quede el archivo.  
Al estar contentos podemos hacer un git add, luego commit y un push como antes.

$ git add README.md 
$ git status
On branch dos
Your branch is up-to-date with 'origin/dos'.
All conflicts fixed but you are still merging.
  (use "git commit" to conclude merge)

Changes to be committed:

	modified:   README.md

$ git commit -m "Dos unido a Uno en rama Dos" 
[dos 955b403] Dos unido a Uno en rama Dos
rho@Sophie:~/Documents/GitHubBranches/Tres$ git push origin dos
Counting objects: 3, done.
Delta compression using up to 8 threads.
Compressing objects: 100% (3/3), done.
Writing objects: 100% (3/3), 347 bytes | 0 bytes/s, done.
Total 3 (delta 1), reused 0 (delta 0)
To git@github.com:Rhodolfo/Ejemplo.git
   d809992..955b403  dos -> dos

Y ya tenemos ambos cambios guardados en la rama dos.


